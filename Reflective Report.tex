% Please do not change the document class
\documentclass{scrartcl}

% Please do not change these packages
\usepackage[hidelinks]{hyperref}
\usepackage[none]{hyphenat}
\usepackage{setspace}
\usepackage{graphicx}
\doublespace

% You may add additional packages here
\usepackage{amsmath}

% Please include a clear, concise, and descriptive title
\title{Reflective Essay}

% Please do not change the subtitle
\subtitle{COMP120 - Reflective Report}

% Please put your student number in the author field
\author{1503048}

\begin{document}

\maketitle

\abstract{In this reflective report, I'll be reviewing my learning while reflecting on my projects and approaches for the first semester.}

\section{Introduction}

The three weakness I explore in this report are: time management, critical evaluation and communication. All of these skills are significant to me for professional practise and have been emphasised during my first semester.

\section{Time Management}

\subsection{Description - What happened? What is being examined?}

One of my weaknesses was time management. With most tasks, it is crucial to follow a strict and thoughtful plan which accounts for all aspects of the project. For me, the relevance of this issue was for not anticipating all aspects of each project.

\subsection{How has this impacted my learning and projects?}

I found that my allocated work time had been misplaced. This not only lead to a low quality of work but also low quantity too.

\subsection{Interpretation - What is relevant about this idea?}

Initially, I noticed that I did not feel confident in my abilities. At the time, I did not know how to overcome this issue which lead to frustration.

\subsection{Outcome- What have I learned and what can be done in the future?}

I now realise my time should have been more focused. I have significantly changed my opinion of how programming works and my knowledge of self teaching has improved. For the future, I aim to make structured session objectives and expected completion times. If the task is not completed within the given time, then I must approach my lecturer or colleagues with my concerns.

\section{Critical Evaluation}

\subsection{Description - What happened? What is being examined?}

Another of my weaknesses was critical evaluation. The ability to critique my own work allows for issues to be discussed and improvements to be made. In addition, the issue is relevant for the objective review of both my peers essays and coding work.

\subsection{How has this impacted my learning and projects?}

Unlike my previous education, these tasks required my own objective evaluation of work quality. This lead to being over critical of my own work, in a way that wasn't constructive.

\subsection{Interpretation - What is relevant about this idea?}

For me, the issue arose from very low confidence. Subsequently, I felt I was not appropriately equipped to offer critical advice. In addition to this, I felt unfamiliar with marking others work with regards to coding practise.

\subsection{Outcome- What have I learned and what can be done in the future?}

I now realise that making reference to project objectives focuses evaluation. In addition, using this template as a checklist proves each topic has been discussed.

\section{Communication}

\subsection{Description - What happened? What is being examined?}

A final weakness was communication, which is relevant for all aspects of any industry. In addition to verbal presentation of ideas, clearly written documents and comments of code are crucial for project work flow. In some aspects of my group work, I failed to engage in discussion which lead to poorly assembled presentations.

\subsection{How has this impacted my learning and projects?}

Documentation and clear comments would have greatly improved my review session feedback. It also would have increased my confidence and knowledge of my own work.

\subsection{Interpretation - What is relevant about this idea?}

For me, the relevance of this aspect is crucial for further employment. A portfolio with evident code comprehension, through appropriate code commenting, and verbal description would greatly improve my employability.

\subsection{Outcome- What have I learned and what can be done in the future?}

Through my use of Github, I know feel confidently using short description to identify issues or variables. To maintain communication skills, I should follow appropriate naming conventions while adding comments during programming sessions.

\section{Communication}

For future learning, and professional practise, I aim to use SMART planning to ensure measureable and achievable standards of work.

%\bibliographystyle{ieeetr}
%\bibliography{PCG_export}

\end{document}