\documentclass{scrartcl}

\usepackage[hidelinks]{hyperref}
\usepackage[none]{hyphenat}
\usepackage{setspace}
\doublespace

\usepackage{amsmath}

\title{Semester Review Report}




\author{DH185421}

\begin{document}

\maketitle

\abstract{This report will be looking at the skills that I have personally been lacking and what next steps I should take towards improving them.}





\section{Key Skill 1 - Communication}

Communication is a key part for a student to use to be able to get and give information regarding the course and work set. For me the most significant aspect was to be able to know what I am doing and know where I am going wrong. Alternatively this is perhaps due to some sort of anxiety I get when asking for help as I don’t want to burden anyone with my problems. Equally this is probably shown by the lack of quality and quantity of my work in the first semester. Having experienced not knowing what to do until it's nearly too late I now feel like I need to ask for help when I need it, additionally I have learned that I need to develop a rapport with my lecturers and classmates so I am not afraid to ask. I have slightly developed my skills in communication by helping and talking about the work with a classmate. This skill will be essential to me as a learner and a practitioner because without it I will not fully understand what is expected of me/work. As a next step I will need to communicate more with my lecturer to make sure I am on task, this can be done by setting up weekly tutorial meetings to measure the work I'm doing and also to upload my work to Github for regular feedback. I can measure this next step by looking at my attendance at these meetings and looking at how many times I commit on Github.

\section{Key Skill 2 - Time Management}
Time management is a key skill especially in programming and writing essays as it cannot be crammed into a single night. For me the most relevant issue was not properly organising my time on task. Subsequently, I noticed that I had started to fall behind on work. Equally this might be due to family matters. Having experienced that having to cram work in under a week is stressful I now feel like I will spend my time more productively. Furthermore, I have learned that even doing a little bit of work each day adds up. I have significantly improved my ability to time manage by doing a little of work each day or at least every other day. This makes me feel more in control of my work is alot less stressful than having to do it all a week before hand in. As a next step I will need to ask for assistance from the ASK team to write out a plan of what work I need to be doing which is achievable and time efficient. I can measure this next step by asking my lecturer to look at the timetable and see if it is acceptable for the course.
\section{Key Skill 3 - Proofing Strategy}

Proofreading your work allows you to see mistakes and improve your work. For me the most significant issue in my written assignment was not proof reading my work and putting time into improving it. Subsequently I knew that this would mean that my reports were not going to get a good grade, equally this is perhaps due to my poor time management and just wanting to get it out the way to spend time on other tasks. Having applied what I haven't done I now know that have to proofread my work if I want to improve my grades. However, I have not yet improved this skill as of yet. This means that until I take time to do this my grades will not represent my true skill. As a next step, I need to proofread my work to find basic mistakes and then I need to find someone I can regularly swap pieces of work and then we read through each others work to pickup on things we've missed. I can measure this success by getting a set partner that is willing to help me.


\end{document}