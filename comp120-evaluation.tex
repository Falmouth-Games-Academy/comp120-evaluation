% Please do not change the document class
\documentclass{scrartcl}

% Please do not change these packages
\usepackage[hidelinks]{hyperref}
\usepackage[none]{hyphenat}
\usepackage{setspace}
\usepackage{graphicx}
\doublespace

% You may add additional packages here
\usepackage{amsmath}

% Please include a clear, concise, and descriptive title
\title{Reflective Report}

% Please do not change the subtitle
\subtitle{COMP120 - Reflective Report}

% Please put your student number in the author field
\author{1507290}

\begin{document}

\maketitle

\abstract{}

\section{Introduction}
Three key skills that have caused me notable difficulty during the course are reading and understanding peers' code, effectively managing my time and priorities, and asking for feedback and clarification when needed.

\section{Reading and Understanding Peers' Code}
Due to a lack of experience in reading and writing code, I have often had difficulty reading, understanding, and picking out the relevant parts when reviewing peers' code.

This is an important skill to develop as it has a number of applications both in education and industry. Reading code is essential in collaborative projects, as it is necessary both to understand the code-base you are contributing to and review the contributions of your peers. It is critical that I develop this skill in preparation for the upcoming group projects.

In order to develop this skill, I will ensure that I read, review and comment on peers' code on GitHub each week. I will assess my progress at the end of the second term by evaluating the difference in time needed to understand and offer appropriate and useful comments on peers' code. 

\section{Time Management and Project Planning}
My perfectionistic nature combined with being out of education and work for a long time has caused me to struggle to manage my time effectively. This has had a significant impact on my productivity, due to spending more time than is needed on projects and increased stress leading to me being less productive.

Time management is critical in both academic and professional contexts. Project planning is necessary for ensuring that deadlines for multiple large projects are able to be met. Is is also essential for effective team work, as it allows members to make progress consistently if they are aware of what other members are working on and when. Improving my time management and project planning skills will allow me to be a reliable team member.

I will tackle this issue by creating a study plan at the beginning of each week to organise my time based on assignment priorities. At the end of the second term, I will assess the success of this strategy by measuring how often I adhered to the plan and whether it allowed me to finish assignments before the deadline. I will also set up a Trello board for sprint planning at the beginning of each project and assess the outcome in a similar manner.

\section{Communication - Asking for Feedback, Help and Clarification}
Anxiety and other personal issues have resulted in me having immense difficulties with communication, specifically with regards to asking for feedback, help and clarification when needed. 

Actively asking for feedback, help, and clarification when needed is an essential skill that a reliable team member must possess. This ensures that the brief and what is expected is properly understood. It prevents time from being wasted on unnecessary work carried out due to misunderstandings of requirements. Asking for help when stuck is also important for maintaining progress, as it prevents excessive periods of being stuck trying to solve insignificant problems.

In order to tackle this problem, I will create a list of questions at the end of each week that I will ask at the next tutorial. I will review this at the end of the second term by assessing whether I have managed to write down and get my questions answered each week. 

\section{Conclusion}


\bibliographystyle{ieeetran}

\end{document}
