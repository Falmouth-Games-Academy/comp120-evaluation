% Please do not change the document class
\documentclass{scrartcl}

% Please do not change these packages
\usepackage[hidelinks]{hyperref}
\usepackage[none]{hyphenat}
\usepackage{setspace}
\usepackage{graphicx}
\doublespace

% You may add additional packages here
\usepackage{amsmath}

% Please include a clear, concise, and descriptive title
\title{Reflective Report}

% Please do not change the subtitle
\subtitle{COMP120 - Reflective Report}

% Please put your student number in the author field
\author{1507290}

\begin{document}

\maketitle

\abstract{}

\section{Introduction}
I have identified three key skills that I have struggled to use and develop throughout the course. I have struggled to read and understand peers' code, as well as finding it difficult to manage my time and priorities. (And another one that I haven't thought of yet)

\section{Reading and Understanding Peers' Code}
When reviewing peers' code, I have often had difficulty reading, understanding, and picking out the relevant parts.

This is an important skill to develop as it has a number of applications both in education and industry. Reading code is essential in collaborative projects, as it is necessary both to understand the code-base you are contributing to and review the contributions of your peers.

My weakness in this area is most likely due to lack of experience reading and writing code. In order to develop this skill, I will ensure that I read, review and comment on peers' code on GitHub each week. I will assess my progress in this skill at the end of the second term. I will know that I have improved by evaluating how long I spend reading a peers' code before I can think of appropriate and useful comments to make on it. 

\section{Time Management and Project Planning}
Another weakness that has affected me throughout the course is the lack of ability to manage my time effectively. This has had a significant impact on my productivity, due to both the obvious reason and due to increased stress leading to me being less productive.

This is an essential skill to develop to ensure success in both academic and professional contexts. Time management and project planning is necessary for ensuring that deadlines for large projects are met, as well as ensuring that other relevant areas are not neglected. This is also an essential skill for effective team work, as it allows team members to make progress consistently if they know what the other members are working on and when. Improving my time management and project planning skills will allow me to be a reliable member of a team.

My perfectionistic nature has likely contributed to my struggle with prioritising tasks and projects, as I do not stop working on a task even though it has been completed. Another cause for this difficulty may be due to having had a long time out of education or work. I will tackle this issue by creating a study plan at the beginning of each week, to organise my time based on assignment priorities. At the end of the second term, I will assess the success of this strategy by measuring how often I managed to stick to the study plan and whether it allowed me to finish assignments before the deadline.

\section{Skill 3}
The third skill that has had a negative impact on my productivity is communication, specifically with regards to asking for feedback, help and clarification when needed.

Actively and regularly asking for feedback, help, and clarification when need is an essential skill that a reliable team member must possess. This way, a team member ensures that they have understood the brief properly and what the other members and product owner are expecting of them. It prevents time from being wasted on unnecessary work carried out due to a misunderstanding of the requirements. Asking for help when stuck is also important for keeping up to date with work on the project, as it allows you to move on instead of been stuck trying to solve an insignificant problem.

In order to tackle this problem, I will create a list of questions at the end of each week that I will ask at the next tutorial. I will review this at the end of the second term by assessing whether I have managed to write down and get my questions answered each week. 

\section{Conclusion}


\bibliographystyle{ieeetran}

\end{document}
