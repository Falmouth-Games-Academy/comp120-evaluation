% Please do not change the document class
\documentclass{scrartcl}

% Please do not change these packages
\usepackage[hidelinks]{hyperref}
\usepackage[none]{hyphenat}
\usepackage{setspace}
\doublespace

% You may add additional packages here
\usepackage{amsmath}

% Please include a clear, concise, and descriptive title
\title{Reflective Report}

% Please do not change the subtitle
\subtitle{COMP120 - Reflective Report}

% Please put your student number in the author field
\author{1507866}

\begin{document}
	
	\maketitle
	
	\abstract{This report will look over the last semester and reflect on what I believe to have been my three main weaknesses}.
	
	\section{Introduction}
	After reviewing my weekly reports for the first semester I feel that my three main weaknesses have been my time management, problems solving skills and communication. 
	
	\section{Time Management}
From experience over the last semester I feel that time management is the skill I need to improve on the most. I've found when I have many assignments set at the same time it can be overwhelming. It would be better to assign time to each one and plan what needs to be done. However I often assign time for an assignment and either not do enough work or I focus too much on other assignments letting the others get behind. This is a problem as it means I'm not spending enough time on some assignments and I'm not planning enough to ensure I get enough work done to stay on schedule.

Previously I have been using Trello boards to plan my Kivy app. In that particular assignment I found Trello very helpful and think a possible way to improve my time management would be to use them for more assignments and make them more detailed. This also helps me to break the assignments down into a series of smaller tasks that make the task seem more doable. The use of Trello boards also links to using sprint plans. For future assignments I am aiming to use more sprint plans.  and stick closely too them to ensure that I get enough work done on each assignment in a week.
	

	
	\section{Problem Solving}
A second weakness from the last semester is my ability to solve problems.  This was initially a problem for the SpaceChem worksheet. On many of the levels I found that my solutions were poor in both the number of elapsed cycles and symbols used.

Problems solving will also effect my coding abilities. For both the university course and afterwards . improving on these skills should help me find better solutions when programming

One way I intend to improve my problem solving skills is to complete  SpaceChem and other problem solving games. I found previously when playing it that levels further on gave me ideas to solve previous ones. Completing such games should help improve my problem solving.  This will be measurable in the games feedback on each level. Another action is to read the Think Like A Programmer textbook which addresses problem solving skills.

I will aim to complete the book and some of the games in the next two months. So I can improve my coding for the next assignments. 
	
	\section{Critiquing Code}
A further a weakness from the past semester is my critical thinking. Currently I find it hard to critique other people's code and when I do I find that I can not sufficiently explain the  issue. I also find it hard to analyze my own code. Improving on this skill will help me improve my code. Also it will help me to give my classmates more constructive feedback in code reviews.

I intend to improve on this by  reviewing other people's code on GitHub more often than just the in class review sessions.  This will help me practice the skill and hopefully improve on it. I will also practice with my own code. I intend to do this by structuring it better and testing more frequently. This will prevent some issues from arising and make them simpler to solve if they do.
	
	\section{Conclusion}
	
	\bibliographystyle{ieeetr}
	\bibliography{comp110_architecture}
	
\end{document}
